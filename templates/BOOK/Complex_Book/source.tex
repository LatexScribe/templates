\documentclass[12pt, oneside]{book}

% Packages for fonts and graphics
\usepackage{lmodern}
\usepackage{graphicx}
\usepackage{hyperref}
\usepackage{url}
\usepackage{xcolor}
\usepackage{amsmath}
\usepackage{amsfonts}
\usepackage{amssymb}
\usepackage{mathrsfs}

% Packages for bibliography management
\usepackage[backend=biber,style=apa]{biblatex}
\addbibresource{references.bib}

% Package for fancy headers and footers
\usepackage{fancyhdr}
\pagestyle{fancy}
\fancyhf{}
\fancyhead[L]{\leftmark}
\fancyhead[R]{\thepage}

% Package for handling table of contents
\usepackage{tocloft}

% Package for handling footnotes and margin notes
\usepackage{marginnote}
\usepackage{footnote}
\makesavenoteenv{tabular}
\makesavenoteenv{table}

% Package for indexing
\usepackage{makeidx}
\makeindex

% Package for handling code listings
\usepackage{listings}
\lstset{basicstyle=\ttfamily,breaklines=true}

% Packages for handling glossaries
\usepackage[toc]{glossaries}
\makeglossaries

% Packages for appendices
\usepackage[toc,page]{appendix}

% Custom chapter and section formatting
\usepackage{titlesec}
\titleformat{\chapter}[display]
  {\normalfont\LARGE\bfseries}
  {\chaptertitlename\ \thechapter}{20pt}{\LARGE}
\titleformat{\section}
  {\normalfont\Large\bfseries}
  {\thesection}{1em}{}

% Custom environments
\usepackage{mdframed}
\newenvironment{summary}[1][Summary]
  {\begin{mdframed}[backgroundcolor=gray!10]\textbf{#1}\par\medskip}
  {\medskip\end{mdframed}}

% Title, author, and date
\title{The Title of Your Complex Book}
\author{Your Name}
\date{\today}

\begin{document}

% Custom Title Page
\begin{titlepage}
    \centering
    \vspace*{1in}
    {\Huge\bfseries The Title of Your Complex Book\par}
    \vspace{1.5in}
    {\Large\itshape Your Name\par}
    \vfill
    {\Large \today\par}
\end{titlepage}

% Dedication
\clearpage
\thispagestyle{empty}
\vspace*{3in}
\begin{center}
    \textit{Dedicated to someone special}
\end{center}

% Abstract
\clearpage
\chapter*{Abstract}
\addcontentsline{toc}{chapter}{Abstract}
This is the abstract of the book. It provides a brief summary of the content and purpose of the book.

% Acknowledgments
\clearpage
\chapter*{Acknowledgments}
\addcontentsline{toc}{chapter}{Acknowledgments}
I would like to thank...

% Preface
\clearpage
\chapter*{Preface}
\addcontentsline{toc}{chapter}{Preface}
This is the preface of the book. It introduces the reader to the book and its contents.

% Foreword
\clearpage
\chapter*{Foreword}
\addcontentsline{toc}{chapter}{Foreword}
This is the foreword of the book. It is typically written by someone other than the author.

% Table of Contents
\clearpage
\tableofcontents

% List of Figures
\clearpage
\listoffigures

% List of Tables
\clearpage
\listoftables

% Glossary
\clearpage
\chapter*{Glossary}
\addcontentsline{toc}{chapter}{Glossary}
\printglossary

% Chapter 1
\clearpage
\chapter{Introduction}
\section{Overview}
This is the overview section of the introduction chapter. You can write an introduction to your book here.

\section{Motivation}
This section can be used to describe the motivation behind writing the book.

\section{Organization}
This section can outline how the book is organized.

% Sample Figure
\begin{figure}[h]
    \centering
    \includegraphics[width=0.5\textwidth]{example-image}
    \caption{An example image}
    \label{fig:example-image}
\end{figure}

% Sample Table
\begin{table}[h]
    \centering
    \begin{tabular}{|c|c|c|}
        \hline
        Column1 & Column2 & Column3 \\
        \hline
        Data1 & Data2 & Data3 \\
        \hline
    \end{tabular}
    \caption{An example table}
    \label{tab:example-table}
\end{table}

% Sample Mathematical Equation
\section{Mathematical Equations}
Here is a sample mathematical equation:
\begin{equation}
    E = mc^2
\end{equation}

\section{Further Discussion}
Further discussion can go here. Use footnotes\footnote{This is a sample footnote.} and margin notes\marginnote{This is a sample margin note.} for additional information.

% Summary Box
\begin{summary}
This is a summary of the key points discussed in this section.
\end{summary}

% Code Listings
\section{Code Listings}
Here is an example of a code listing:
\begin{lstlisting}[language=Python, caption=Sample Python Code]
def hello_world():
    print("Hello, world!")
\end{lstlisting}

% Citations
\section{Citations}
Cite references using \textcite{exampleBook} and \autocite{exampleBook}.

% More chapters...
\chapter{Another Chapter}
This is another chapter with more content.

\chapter{Yet Another Chapter}
Here is yet another chapter.

\chapter{Conclusion}
\section{Summary}
This is the summary section of the conclusion chapter.

\section{Future Work}
This section can be used to describe future work or directions.

% Appendices
\clearpage
\chapter*{Appendices}
\addcontentsline{toc}{chapter}{Appendices}

\begin{appendices}
\chapter{Appendix A}
\section{Additional Information}
This is the first appendix.

\chapter{Appendix B}
\section{More Information}
This is the second appendix.
\end{appendices}

% Index
\clearpage
\printindex

% References
\clearpage
\printbibliography

\end{document}
